Katika neema ya Yesu, 
nimeokolewa, 
Nilipotea dhambini,  
nilikuwa kipofu rohoni 
